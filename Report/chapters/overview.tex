\chapter{Implementation Overview}
\label{sec:implementation-overview}

The goal is to develop a secure Password Manager, where passwords are stored in a secure and removable USB device, that is based on SECube. Different challenges need to be solved in order to achieve this goal: from the SECube's firmware to the development of a Chromium Extension. \bigskip

The main problem is the communication between the Chromium Browser and the SECube: due to the high level of restrictions imposed by any modern web browser, indeed they act as a big and complex sandboxes for good reasons, it's nearly impossible to have a direct connection between the Extension and any physical device connected to the Host PC. \bigskip

Because of this, a third actor needs to be introduced: a middleware that is capable of being both interfaced with any (Chromium-based) browser and able to communicate with the SECube device. Thus, the middleware is a third software that is intended to be installed and run on the host PC continuously (like a service on Windows or a daemon on Linux, or executed by the user when he needs it, as he prefers). \bigskip

\section{The solution}

\begin{figure}[H]
	\centering
	\includegraphics[width=0.9\linewidth]{images/overview}
	\caption{How the three actors interact between each other}
	\label{fig:overview}
\end{figure}

Due to all these actors, developing a secure Password Manager becomes very complicated. It is very important, in cases like this, to be pretty organized and try to be systematic in developing all the needed software, without trying to reinvent the wheel. \bigskip

The three components have been developed completely independent from each others, trying to move all the possible logic and sensitive data to the lowest level (i.e. the Host Middleware and the SECube): this is due to the fact that these two actors are the most secure while the Extension is the most exposed one to possible threats. \bigskip

The communication between the Extension and the Middleware happens by means of HTTPs: it is a secure version of HTTP based on TLS, allowing to have a complete communication between the two parties. The communication between the Middleware and the SECube happens by means of a USB connection, and the communication is encrypted. 
\chapter{Introduction}
Passwords are a key part of everyday life and to be secure they must be complex. But this complexity is in conflict with the ability to remember them. Another problem arises if we think about a secure place where to store our passwords. Is a server safe enough? Better a text file? What about a database?
The best solution to this critical problem is for sure a password manager. A physical password manager is even better! This is the motivation that led to the creation of this project.

The main focus is security, followed by user experience (exploiting UX and UI). The project is aimed at the development of a password manager that can be used by any user, regardless of his or her experience with the software. The development is done from scratch, starting from the firmware, going through the host middleware and finally to the extension (icon shown in \autoref{fig:extension-icon}).

In \autoref{sec:background} is discussed the Background of the project, in order to explain to everyone what is a password manager, what is the purpose of it and what is the state of the art. In \autoref{sec:implementation-overview} is given a brief overview of the project implementation. In \autoref{sec:implementation-details} all implementation details are explained. This chapter is very technical and tries to unfold all the decisions taken in the making process of the project.
In \autoref{sec:results} all results are illustrated, focusing also on known issues and future work. In \autoref{sec:conclusions} conclusions are presented. In the \autoref{usermanual} there is the User Manual, in order to explain to more expert and brave users how to eventually modify the whole project.

\vspace*{1cm}
\begin{figure}[H]
	\centering
	\includegraphics[width=0.27\linewidth]{images/extension/password.png}
	\caption{Icon of the Chrome Extension}
	\label{fig:extension-icon}
\end{figure}
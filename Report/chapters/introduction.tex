\chapter{Introduction}
%% DELETE THE TEXT BELOW
%In this first chapter we expect you to introduce the project explaining what the project is about, what is the final goal, what are the topics tackled by the project, etc.\newline The introduction must not include any low-level detail about the project, avoid sentences written like: we did this, then this, then this, etc.\newline It is strongly suggested to avoid expressions like `We think`, `We did`, etc...it is better to use impersonal expressions such as: `It is clear that`, `It is possible that`, `... something ... has been implemented/analyzed/etc.` (instead of `we did, we implemented, we analyzed`).\newline In the introduction you should give to the reader enough information to understand what is going to be explained in the remainder of the report (basically, expanding some concept you mentioned in the Abstract) without giving away too many information that would make the introduction too long and boring.\newline Feel free to organize the introduction in multiple sections and subsections, depending on how much content you want to put into this chapter.

%Remember that the introduction is needed to make the reader understand what kind of reading he or she will encounter. Be fluent and try not to confuse him or her.
%The introduction must ALWAYS end with the following formula: The remainder of the document is organized as follows. In Chapter 2, ...; in Chapter 3, ... so that the reader can choose which chapters are worth skipping according to the type of reading he or she has chosen.


Passwords are a key part of everyday life and to be secure they must be complex. But this complexity is in conflict with the ability to remember them. Another problem arises if we think about a secure place where to store our passwords. Is a server safe enough? Better a text file? What about a database?
The best solution to this critical problem is for sure a password manager, even better a physical password manager. This is the motivation that led to the creation of this project.

The main focus is security, followed by user experience (exploiting UX and UI). The project is aimed at the development of a password manager that can be used by any user, regardless of his or her experience with the software. The development is done from scratch, starting from the firmware, going through the host middleware and finally to the extension (icon shown in \autoref{fig:extension-icon}).

In \autoref{sec:background} is discussed the Background of the project, in order to explain to everyone what is a password manager, what is the purpose of it and what is the state of the art. In \autoref{sec:implementation-overview} is given a brief overview of the project implementation. In \autoref{sec:implementation-details} all implementation details are explained. This chapter is very technical and tries to unfold all the decisions taken in the making process of the project.
In \autoref{sec:results} all results are illustrated, focusing also on known issues and future work. In \autoref{sec:conclusions} conclusions are presented. In the \autoref{usermanual} there is the User Manual, in order to explain to more expert and brave user how to eventually modify the whole project.

\vspace*{1cm}
\begin{figure}[H]
	\centering
	\includegraphics[width=0.4\linewidth]{images/extension/password.png}
	\caption{Icon of the Chrome Extension}
	\label{fig:extension-icon}
\end{figure}
\chapter{API}
\label{api}


\section{Host Library L1}
In order to allow to use all functionalities that are provided by the firmware developed for the SECube, the L1 library has been enhanced with additional functions. Even if these methods are used only by the Python Middleware via wrappers, they can be directly used by a custom user Host application.

The following list contains the exhaustive description of all added methods for the Secure Password Manager. Each one of them can be used only after a correct authentication via the L1 \texttt{L1Login} method. All APIs are defined into a custom .cpp file called \texttt{L1\_sepass.cpp} while the signatures are available at \texttt{L1.h}.\newline\newline
Most of the functions are simply used to fill the buffer for the command request and to parse the response, since the application logic has been moved to the firmware itself to increase the security. An example of usage of all the functions below is available at \texttt{examples/manual\_pass\_management.cpp}.

\subsection{Type definition}
In order to simply the interactions with the L1 Secure Password Manager APIs, a custom structure called \texttt{se3Pass} has been defined:
\begin{lstlisting}[language=C++,breaklines=true]
	typedef struct se3Pass_ {
		uint32_t id;
		uint16_t hostSize;
		uint16_t userSize;
		uint16_t passSize;
		std::string host;
		std::string user;
		std::string pass;
	} se3Pass;
\end{lstlisting}

This structure is able to manage as \texttt{string} the hostname, username and password of a Secure Password Manager record. By the way, the total bytes for all fields must not be greater than 4094 byte in total, due to a firmware limitation. This has meaning for both \texttt{L1SEAddPassword} and \texttt{L1SEModifyPassword} methods.\newline\newline
To increase the flexibility of the search functionality, the results can be filtered from the firmware accordingly to:
\begin{itemize}
	\item Hostname
	\item Username
\end{itemize}

For simplicity, an \texttt{enum} creates a correspondence between the numerical value and it's intrinsic information, provided by the name.
\begin{lstlisting}[language=C++,breaklines=true]
	typedef enum se3_list_filter_type_ {
		NO_FILTER = 0,
		HOST_FILTER = 1,
		USER_FILTER = 2
	} se3_list_filter_type;
\end{lstlisting}

\subsection{L1SEGetAllPasswords}
This function exports the list of all passwords in plain text from the SEcube flash memory. In this case, the list is returned as a vector of \texttt{se3Pass} elements. In order to call this function, a vector of type \texttt{se3Pass} must be initialize (e.g. \texttt{std::vector<se3Pass> passList;}).\newline\newline
The function has also a boolean return, that in case of error is set to \texttt{false}.

\begin{lstlisting}[language=C++,breaklines=true]
	bool L1SEGetAllPasswords(std::vector<se3Pass>& passList);
\end{lstlisting}

\subsection{L1SEGetAllPasswordsByUserName}
This function exports the list of all passwords in plain text from the SEcube flash memory filtered by the username, passed as a \texttt{string} parameter. In this case, the list is returned as a vector of \texttt{se3Pass} elements. In order to call this function, a vector of type \texttt{se3Pass} must be initialize (e.g. \texttt{std::vector<se3Pass> passList;}).\newline\newline
The output is the same of the \texttt{L1SEGetAllPasswords} function, but filtered by username.\newline\newline
The function has also a boolean return, that in case of error is set to \texttt{false}.

\begin{lstlisting}[language=C++,breaklines=true]
	bool L1SEGetAllPasswordsByUserName(std::vector<uint8_t> username, std::vector<se3Pass>& passList);
\end{lstlisting}

\subsection{L1SEGetAllPasswordsByHostName}
This function exports the list of all passwords in plain text from the SEcube flash memory filtered by the hostname, passed as a \texttt{string} parameter. In this case, the list is returned as a vector of \texttt{se3Pass} elements. In order to call this function, a vector of type \texttt{se3Pass} must be initialize (e.g. \texttt{std::vector<se3Pass> passList;}).\newline\newline
The output is the same of the \texttt{L1SEGetAllPasswords} function, but filtered by hostname.\newline\newline
The function has also a boolean return, that in case of error is set to \texttt{false}.

\begin{lstlisting}[language=C++,breaklines=true]
	bool L1SEGetAllPasswordsByHostName(std::vector<uint8_t> hostname, std::vector<se3Pass>& passList);
\end{lstlisting}

\subsection{L1SEAddPassword}
The \texttt{L1SEAddPassword} allows to create a new password record by providing:
\begin{itemize}
	\itemsep0sp
	\item ID
	\item Length of the hostname
	\item Length of the username
	\item Length of the password
	\item Hostname
	\item Username
	\item Password
\end{itemize}

\begin{lstlisting}[language=C++,breaklines=true]
	bool L1SEAddPassword(uint16_t pass_id, uint8_t *host_data, uint16_t host_len, uint8_t *user_data, uint16_t user_len, uint8_t *pass_data, uint16_t pass_len);
\end{lstlisting}

The function returns true if the record has been written into the memory of the SECube, in case of error \texttt{false} is returned.

\subsection{L1SEGetPasswordById}
Given a numerical password id, the password record is stored into the \texttt{se3Pass} structure passed as reference. The structure, before calling the method can be initialized as: \texttt{se3Pass searchedById;}.
The function has also a boolean return, that in case of error is set to \texttt{false}.

\begin{lstlisting}[language=C++,breaklines=true]
	bool L1SEGetPasswordById(uint32_t pass_id, se3Pass& password);
\end{lstlisting}

\subsection{L1SEModifyPassword}
Given a numerical password id, the password record is modified accordingly to the new provided \texttt{se3Pass}.
The function has also a boolean return, that in case of error is set to \texttt{false}.
\begin{lstlisting}[language=C++,breaklines=true]
	bool L1SEModifyPassword(uint32_t pass_id, se3Pass& password);
\end{lstlisting}

\subsection{L1SEDeletePassword}
Given a numerical password id, the password with that identified is deleted. If the password is not found or some errors occur, \texttt{false} is returned.
\begin{lstlisting}[language=C++,breaklines=true]
	bool L1SEDeletePassword(uint32_t pass_id);
\end{lstlisting}

\subsection{L1SEGenerateRandomPassword}
Generate a random password with the defined length and with the possibility to include uppercase, special characters and numbers. Via three variables it's possible to enable or disable the presence of a specific character set:
\begin{itemize}
	\itemsep0sp
	\item \texttt{enable\_upper\_case}: uppercase characters
	\item \texttt{enable\_special\_chars}: special characters
	\item \texttt{enable\_numbers\_chars}: symbols
\end{itemize}
The \texttt{pass\_len} parameter allows to define the password dimension, while the result is returned as an array of \texttt{uint8\_t}.

\begin{lstlisting}[language=C++,breaklines=true]
	bool L1SEGenerateRandomPassword(uint16_t pass_len, uint8_t enable_upper_case, uint8_t enable_special_chars, uint8_t enable_numbers_chars, std::shared_ptr<uint8_t[]> generated_pass);
\end{lstlisting}

It's important to mention that all the complexity of the generation is moved into the SECube to increase the security. Moreover, if a character set is selected, at least one char of that type must be included into the output random password.



\section{Middleware HTTPs' API}

\subsection{/api/v0/time}
Used to work with the timestamp. The timestamp is an integer in seconds. The supported HTTP methods are:

\begin{itemize}
    \item \texttt{GET}: returns the current timestamp.
\end{itemize}

\subsection{/api/v0/devices}
Used to work with the devices. It allows to obtain all the connected boards, in particular for each device the ID, Name and Serial are returned. The API is currently not used by the Extension because it' supposed that only one device at a time is connected. The supported HTTP methods are:

\begin{itemize}
    \item \texttt{GET}: returns the list of devices.
\end{itemize}

\subsection{/api/v0/device/\{id\}/sessions}
Used to manage sessions. The supported HTTP methods are:

\begin{itemize}
    \item \texttt{GET}: allow to know if the cookie attached to the request represents a valid session or not.
    \item \texttt{POST}: creates a new session. The \textit{PIN} and the \textit{timestamp} parameters are mandatory. The \textit{timestamp} parameter is an integer in seconds.
    \item \texttt{DELETE}: forces to invalidate the session attached to the cookie.
\end{itemize}

\subsection{/api/v0/device/\{id\}/generate}
Used to generate a new password using the exposed funcionality of the board. The supported HTTP methods are:

\begin{itemize}
    \item \texttt{GET}: allows to obtain a new randomly generated password. The optional parameters are:
        \subitem \texttt{length}: the length of the password. The default value is 64.
        \subitem \texttt{upper}: boolean value that indicates if the password must contain uppercase letters. The default value is 1. Can be 0.
        \subitem \texttt{special}: boolean value that indicates if the password must contain special characters. The default value is 1. Can be 0.
        \subitem \texttt{numbers}: boolean value that indicates if the password must contain numeric characters. The default value is 1. Can be 0.
\end{itemize}

\subsection{/api/v0/device/\{id\}/passwords}
Used to manage passwords. The supported HTTP methods are:

\begin{itemize}
    \item \texttt{GET}: allows to obtain the list of passwords. It supports the \texttt{hostname} parameter to filter the list of passwords by hostname. The \texttt{hostname} parameter is a string. Each password is represented by a JSON object with the following fields:
        \subitem \texttt{hostname}: the hostname of the password.
        \subitem \texttt{password}: the password.
        \subitem \texttt{username}: the username.
        \subitem \texttt{id}: the ID of the password.
    \item \texttt{POST}: allows to add and store in the board a new password. The parameters must be passed via the body in the form of a JSON object. The mandatory parameters are:
        \subitem \texttt{hostname}: the hostname of the password.
        \subitem \texttt{password}: the password.
        \subitem \texttt{username}: the username.
\end{itemize}

\subsection{/api/v0/device/\{id\}/password/\{id\}}
Allows to manage a single password. The supported HTTP methods are:

\begin{itemize}
    \item \texttt{GET}: allows to obtain the password record. The password is represented by a JSON object with the following fields:
        \subitem \texttt{hostname}: the hostname of the password.
        \subitem \texttt{password}: the password.
        \subitem \texttt{username}: the username.
        \subitem \texttt{id}: the ID of the password.
    \item \texttt{DELETE}: allows to delete the password.
    \item \texttt{PUT}: allows to update the password. The parameters must be passed via the body in the form of a JSON object, as the one to add a new password. The mandatory parameters are:
        \subitem \texttt{hostname}: the hostname of the password.
        \subitem \texttt{password}: the password.
        \subitem \texttt{username}: the username.
\end{itemize}
\chapter{User Manual}
\label{usermanual}


In the user manual you should explain, step-by-step, how to reproduce the demo that you showed in the oral presentation or the results you mentioned in the previous chapters.\\ If it is necessary to install some toolchain that is already well described in the original documentation (i.e., Espressif's toolchain for ESP32 boards or the SEcube toolchain) just insert a reference to the original documentation (and remember to clearly specify which version of the original documentation must be used). There is no need to copy and paste step-by-step guides that are already well-written and available.\\The user manual must explain how to re-create what you did in the project, no matter if it is low-level code (i..e VHDL on SEcube's FPGA), high-level code (i.e., a GUI) or something more heterogeneous (i.e. a bunch of ESP32 or Raspberry Pi communicating among them and interacting with other devices).  

\section{Host Middleware}

In this section, we will describe how to build and run the host middleware, both on Linux and Windows. The build process is not necessary because a ready-to-run executable will be provided both for Linux and Windows. However, if there are problems in executing them, the build process can be used to try to launch the executable. 

\begin{warning}
Because of all the dependencies and operations to do to build, some problems may occurr. This section will try to indicate all the necessary software that is required, but unfortunately the successful build of the host middleware is not guaranteed because of the heterogeneous nature of our computers.
\end{warning}

\subsection{Linux}
Luckily, on Linux the build process is pretty straightforward. The only thing that needs to be done is to install the necessary software. The following is a list of software that is required to build the host middleware.

\begin{itemize}
    \item \textbf{Python 3.9.x} (tested with 3.9.7). Check that your PATH environment variable points to the Python executable \textit{python3}.
    \item \textbf{pip 20.3.4} (tested with 20.3.4). Check that your PATH environment variable points to the pip executable \textit{pip} and refers to the correct Python version.
    \item \textbf{gcc 11.x} {tested with 11.2.0}. Check that your PATH environment variable points to the gcc executable \textit{gcc}.
    \item \textbf{g++ 11.x} {tested with 11.2.0}. Check that your PATH environment variable points to the g++ executable \textit{g++}.
    \item \textbf{GNU Make} {tested with GNU Make 4.3}. Check that your PATH environment variable points to the GNU Make executable \textit{make}.
    \item \textbf{git} Check that your PATH environment variable points to the git executable \textit{git}.
\end{itemize}

Here a few steps to build the host middleware if all the required software is installed correctly. Note, it can change depending on the used linux distribution. It may requre further steps to install the dependencies.

\begin{lstlisting}[language=bash,caption={bash version}]
$ git clone https://github.com/SEcube-Project/Browser-Password-Manager.git
$ cd Browser-Password-Manager/HostMiddleware

# To build the shared library
$ make clean
$ make -j4 lib.so

# To run the scrypt as-is (include build of lib.so)
$ make clean
$ make -j4 run

# To compile, obfuscate and pack into a single executable
$ make clean
$ make -j4 dist
$ ./BPMMiddleware

\end{lstlisting}


\subsection{Windows}
Unfortunately, on Windows there is a lot more work to do.

\subsubsection{Python}
Python 3.9.x is needed. If you are not sure it's installed on your system, try to launch a Powershell console and type \texttt{python --version}. If you get \texttt{Python 3.9.0} or similar means that Python is installed. Otherwise, you need to install it. \\

\begin{warning}
\textbf{ATTENTION}: if the Windows Store opens up, close it! You need to install it in the \textit{classic way} otherwise strange things will happens later one. In the same way, if you installed python from the Windows Store, unistall it and download it from \href{https://www.python.org/downloads/release/python-390/}{the official Python webpage}.
\end{warning}

\newpage
\begin{warning}
\textbf{ATTENTION}: if python is not available from the Powershell after manual installation, try to reboot. If it's still not available, you need to manually specify the Python's executable path. Start menù, type Python, right-click and select "Open File location". Most likely it will head you to the Start Menu Shortcuts, so right-click again on the Python 3.9 folder and click on "Open file location". Select the path and copy in the clipboard. 

In the start menu, search for "environment" and click "Edit the system environment variable". Click on "environment variables" button, select "Path", click "Edit", click "New" and paste the path you previously copied.

Confirm and close everthing, the Powershell too. Open it again and check if now python is available. 
\end{warning}

\subsubsection{C++ Compiler - Buildtools}
Head to the start menù and look for \texttt{x64 Native Tools Command Prompt for VS 2022} (if you are on a 32 bit system, look for \texttt{x64 Native Tools Command Prompt for VS 2022}). Open it, and a terminal emulator will show up. Type \texttt{cl}. If you get \textit{'cl' is not recognized as an internal or external program...} means that something is missing, otherwise you will get the following message and it means that the compiler is installed. Same reasoning must undergo with the \texttt{link} command.

\begin{lstlisting}[language=bash,caption={bash version}]
C:\Program Files\Microsoft Visual Studio\2022\Community>cl
Microsoft (R) C/C++ Optimizing Compiler Version 19.32.31329 for x64
Copyright (C) Microsoft Corporation.  All rights reserved.

usage: cl [ option... ] filename... [ /link linkoption... ]

C:\Program Files\Microsoft Visual Studio\2022\Community>link
Microsoft (R) Incremental Linker Version 14.32.31329.0
Copyright (C) Microsoft Corporation.  All rights reserved.

 usage: LINK [options] [files] [@commandfile]

   options:

      /ALIGN:#
      /ALLOWBIND[:NO]
      /ALLOWISOLATION[:NO]
      /APPCONTAINER[:NO]
      /ASSEMBLYDEBUG[:DISABLE]
      /ASSEMBLYLINKRESOURCE:filename
      /ASSEMBLYMODULE:filename
      /ASSEMBLYRESOURCE:filename[,[name][,PRIVATE]]
      /BASE:{address[,size]|@filename,key}
      /CLRIMAGETYPE:{IJW|PURE|SAFE|SAFE32BITPREFERRED}
      /CLRLOADEROPTIMIZATION:{MD|MDH|NONE|SD}
      /CLRSUPPORTLASTERROR[:{NO|SYSTEMDLL}]
      /CLRTHREADATTRIBUTE:{MTA|NONE|STA}
      /CLRUNMANAGEDCODECHECK[:NO]
      /DEBUG[:{FASTLINK|FULL|NONE}]
      /DEF:filename
      /DEFAULTLIB:library
      /DELAY:{NOBIND|UNLOAD}
      /DELAYLOAD:dll
      /DELAYSIGN[:NO]
      /DEPENDENTLOADFLAG:flag
      /DLL
      /DRIVER[:{UPONLY|WDM}]
      /DYNAMICBASE[:NO]
      /EMITVOLATILEMETADATA[:NO]
(press <return> to continue)

\end{lstlisting}

If something is missing (or the Visual Studio's Command Prompt Tool is not available), Visual Studio must be installed. Go to \href{https://visualstudio.microsoft.com/downloads/}{https://visualstudio.microsoft.com/downloads/} and download Visual Studio Community edition. Once the installer is downloaded, launch it, select \textit{Visual Studio Community 2022} (click on Modify if Visual Studio is already installed) and select \textit{Desktop Development with C++}. The following parts must be installed:

\begin{itemize}
    \item MSVC v143 - VS 2022 C++ x64/x86 build Tools
    \item Windows 10 SDK 
    \item C++/CLI support for v143 build Tools
    \item C++ Modules for v143 build tools
    \item C++ Clang tools for Windows 
\end{itemize}

Repeat from the beginning, be sure that the \textit{Visual Studio's Command Prompt} is installed and the compiler is available. 

\subsubsection{How to build}
Now everything should be installed. Open the Visual Studio's Command Prompt, head to the HostMiddleware folder (with the CD command) and type \texttt{compile\_win.bat}. It will compile everything and pack into a single BPMMiddleware.exe executable that eventually you can run, if everything went fine.

